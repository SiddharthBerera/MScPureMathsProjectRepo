\chapter{Supporting Material for Chapter 1}

\section{Singular Homology Preliminaries}\label{MorseTheoryAppendix}
\begin{definition}[Standard $m$-simplex] \label{def:std-simplex}
The standard $m$--simplex is the compact set
$$ \Delta^{m}\;=\;\bigl\{(t_1,\dots,t_m)\in\mathbb R^{m}\,\big|\,t_i\ge 0,\;\sum_{i=1}^{m} t_i \le 1\bigr\}.$$ 
For $m=1$, this corresponds to the unit interval $[0,1]$.\\
For $m=2$, this corresponds to the right angled triangle with corners $(0,0), (1,0), (0,1)$.
\end{definition}

\begin{definition}[Singular $m$--chains]
Fix a coefficient ring (usually a field) $k$ and a topological space $X$.
\begin{itemize}
\item A singular $m$--simplex in $X$ is a continuous map
      $\sigma:\Delta^{m}\longrightarrow X$, where $\Delta^{m}$ is the standard
      $m$–simplex (Definition~\ref{def:std-simplex}). Intuitively for say $m=2$, this is the function which sends the 2-$d$ simplex to a crumpled triangle in the manifold $X$.

\item The free $k$–module generated by all singular $m$–simplices is
      written
      \[C_{m}(X;k):= \Bigl\{\sum_{i=1}^{\ell} a_{i}\,\sigma_{i}\;\Bigm|\;a_{i}\in k,\;\sigma_{i}:\Delta^{m}\to X\text{ singular $m$–simplices},\;\ell<\infty\Bigr\}.\]
      The sum is purely algebraic and so distinct $m$-simplices form orthogonal directions in this vector space. An element of $C_{m}(X;k)$ is called a singular $m$--chain. 
      The sum is finite, so every chain is a formal linear combination
      of only finitely many simplices.

\item The boundary operator
      $\partial_m:C_{m}(X;k)\longrightarrow C_{m-1}(X;k)$ is defined on
      a simplex by the alternating face formula
      \begin{equation}\label{boundaryOperator}
          \partial_m \sigma
        =\sum_{j=0}^{m}(-1)^{j}\,
          \sigma\!\bigl|_{[v_0,\dots,\widehat{v_j},\dots,v_m]},
      \end{equation}
      and this extends to a singular $m$-chain by summing the resulting $m-1$-simplices in the ring $k$ over the $m$-simplices in the original chain.  This results in,
      $\partial_{m-1}\circ\partial_m=0$.

\item The sequence
      \(
        \cdots\;\xrightarrow{\partial_{m+1}} C_{m}(X;k)
                \;\xrightarrow{\partial_{m}}   C_{m-1}(X;k)
                \xrightarrow{\partial_{m-1}}\;\cdots
      \)
      is the singular chain complex of $X$ with coefficients in $k$.
\end{itemize}
\end{definition}
\begin{remark}[Intuition for the alternating signs]
For a $2$–simplex (triangle embedded in $\mathbb{R}^2$) with ordered vertices $(v_0,v_1,v_2)$
the boundary formula
\[
\partial_2\sigma
  =\;
    \sigma|_{[v_1,v_2]}
  \;-\;
    \sigma|_{[v_0,v_2]}
  \;+\;
    \sigma|_{[v_0,v_1]}
\]
takes three oriented edges with signs
$+\, -\, +$.
Starting at $v_0$ this moves counter-clockwise once around the
triangle: first along $[v_1,v_2]$, then back along $[v_2,v_0]$ (hence the
minus sign), and finally along $[v_0,v_1]$.
Thus $\partial_2\sigma$ is the “loop” that encloses the triangle.

The alternating signs are chosen precisely so that when you take the
boundary again every edge appears twice with opposite orientation,
hence $\partial_{1}\!\bigl(\partial_{2}\sigma\bigr)=0$.
The same idea works in all dimensions: faces of a simplex come in
cancelling pairs, so $\partial_{m-1}\circ\partial_{m}=0$.
\end{remark}


\begin{definition}[Simplicial Complex]
Let \(V\) be a finite vertex set. A \emph{simplicial complex} \(\mathcal{K}\) on \(V\) is a collection of subsets of \(V\) such that:
\begin{enumerate}
    \item \textbf{Closed under taking faces:} If \(\sigma \in \mathcal{K}\) and \(\tau \subseteq \sigma\), then \(\tau \in \mathcal{K}\).
    \item \textbf{Contains singletons:} For every vertex \(v \in V\), the singleton \(\{v\} \in \mathcal{K}\).
\end{enumerate}
The elements of \(\mathcal{K}\) are called \emph{simplices}, and the dimension of a simplex \(\sigma\) is defined as \(\dim(\sigma) = |\sigma| - 1\), where \(|\sigma|\) is the number of elements in \(\sigma\). A \(k\)-simplex is a simplex of dimension \(k\), and the collection of all simplices of dimension at most \(k\) forms the \(k\)-skeleton of the simplicial complex.
\end{definition}


\begin{definition}[Orientation]
    Let $\sigma=[v_0,\dots,v_m]$ be an $m$–simplex with vertex set $\{v_0,\dots,v_m\}$. Two orderings of the vertices determine the same orientation iff they differ by an even permutation.
    Replacing the ordering by an odd permutation reverses the orientation. Hence an orientation of $\sigma$ is an equivalence class of orderings modulo even permutations.\\
    An orientation of a finite simplicial complex $K$ is a choice of orientation for every simplex such that the induced orientations on any common face agree, i.e.\ differ by an even permutation.
    With this choice, the boundary operator
        \[
          \partial_m[v_0,\dots,v_m]
          \;=\;\sum_{j=0}^{m}(-1)^{j}\,[v_0,\dots,\widehat{v_j},\dots,v_m]
        \]
        satisfies $\partial_{m-1}\circ\partial_m=0$ for chains over
        $\mathbb Z$.
\end{definition}

\begin{example}[A 2-boundary in the Tetrahedron]
Consider the tetrahedron $\Delta^3$ with ordered vertices $V = (v_0,v_1,v_2,v_3)$ and face set of singular $2$-simplices, $F = \{ \sigma_{012}, \sigma_{013}, \sigma_{023}, \sigma_{123} \}$, embedded in $\mathbb{R}^3$. Let us consider a chain in the free $\mathbb{Z}$-module on $\Delta^3$ generated by singular $2$-simplices, namely $C_{2}(\Delta^3;\mathbb{Z})$, which is the set of finite formal linear combinations of singular $2$-simplices. An example of such a chain is,
$$c = \sigma_{123} - \sigma_{023} + \sigma_{013} - \sigma_{012} \in C_{2}(\Delta^3;\mathbb{Z}).$$
In the alternating face formula for the boundary operator, Equation (\ref{boundaryOperator}), we have that
\begin{align*}
\partial_3 \big(\Delta^3\big) = \partial_3 (\sigma_{0123}) = \sum_{j=0}^{3}(-1)^{j}\,\sigma\!\bigl|_{[v_0,\dots,\widehat{v_j},\dots,v_4]} &= \sigma_{123} - \sigma_{023} + \sigma_{013} - \sigma_{012} = c.
\end{align*}
However, because consecutive boundary operators compose to zero ($\partial_{m-1}\circ \partial_{m}=0$) we have $$0=\partial_2(\partial_3 (\sigma_{0123})) = \partial_2 (c).$$ Hence  $c \in Z_2(\Delta^3;\mathbb{Z})$ i.e. is a $2$-cycle for $\Delta^3$.

Intuitively, an m-chain is an oriented sum of “m-dimensional edges” (simplices). The sign records which way we traverse that face: reversing the vertex order multiplies the simplex by -1. Since $c$ is a $2$-cycle its boundary vanishes meaning that every oriented 1-simplex occurs equally
often with positive and negative orientation, so when adding them “tip to tail” they cancel out and the chain starts where it ended, thus we have the name “cycle”.

\noindent We can in fact say more, $c$ is not only a cycle but also a boundary. Indeed, $$\partial_3 \big(\Delta^3\big) = c.$$
As stated in Definition (\ref{cycles_n_boundaries}), from an algebraic viewpoint a boundary is an $m$-chain that lies in the image of the boundary operator; geometrically, however, it is an $m$-cycle that \emph{exactly bounds} some $(m+1)$-simplex (here the 3-simplex \(\sigma_{0123}\)).  Every boundary is therefore a cycle (\(\partial_{m-1}\!\circ\partial_{m}=0\)), but not every cycle is a
boundary: a cycle may “loop around a hole” without enclosing any higher-dimensional simplex.
\end{example}

\begin{remark}
For a path-connected topological manifold $X$, the $0$-th order Homology group for $X$, $H_0(X;k)$, is isomorphic to $k$. This is because the set of $0$ chains, $C_0(X;k)$, is simply all finite formal linear combinations of points in $X$ with coefficients in $k$ and the set of $0$-cycles is all $0$-chains due to $0$-simplices having no boundary. $B_0 = im(\partial_1)$ is the subgroup of $C_0$ that is generated by the set of formal differences of points that have a path passing through both (endpoints of an oriented 1-simplex), which is the set of formal linear combinations of points in $X$ with the sum of coefficients, $0$ according to addition in $k$. Two $0$-chains, $c_1$ and $c_2$, are thus in the same equivalence class under $\sim_0$ iff they have the same total sum of coefficients and so the equivalence classes are just the possible totals, namely $k$.
\end{remark}

\begin{lemma}\label{lem:A1}
For topological space $X$ and a coefficient field $k$, the Homology groups $H_m(X;k)$ are vector spaces.
\end{lemma}
\begin{proof}
The Chain groups $C_m(X;k)$ from a vector space over $k$ as they inherit coefficient addition form a field and they are just finite formal linear combinations of $m$-simplices - the set of distinct singular $m$-simplices form the basis.

By definition, the map $\partial_m:C_{m}(X;k)\longrightarrow C_{m-1}(X;k)$ is linear over singular $m$-simplices. Thus, $Z_m = \ker(\partial_m)$ and $B_m=\operatorname{im}(\partial_{m+1})$ are linear subspaces of $C_m(X;k)$.

Now if $V$ is a vector space and $W\leq V$ a subspace, the quotient $V/W$ inherits addition and scalar multiplication from V; hence it is again a vector space over the same field $k$. Taking $V=Z_m$ and $W=B_m$, the result follows.
\end{proof}